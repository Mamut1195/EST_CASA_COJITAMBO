\documentclass{article}
\usepackage[utf8]{inputenc} % Codificación UTF-8 para caracteres latinos
\usepackage[T1]{fontenc} % Codificación de la fuente para caracteres acentuados
\usepackage[spanish]{babel} % Idioma español
\usepackage{xcolor}
\usepackage{tikz}
\usepackage{graphicx}
\usepackage{pagecolor} % Paquete para cambiar el color de fondo de la página
\usepackage{enumitem}
\usepackage{tabularx}

% Definir un nuevo color personalizado
\definecolor{azulportada}{RGB}{0,25,51} % Color azul RGB

% Paquetes básicos
\usepackage{amsmath} % Paquete matemático
\usepackage{amsfonts} % Fuentes matemáticas
\usepackage{amssymb} % Símbolos matemáticos
\usepackage{graphicx} % Inclusión de gráficos
\usepackage{float} % Control de posición de figuras y tablas
\usepackage{hyperref} % Enlaces hipertexto
\usepackage{geometry} % Configuración de márgenes y tamaño del papel


% Definir información de la portada
\title{Título del Documento}
\author{Jonnathan Torres Vázquez}
\date{\today}

% Configuración de márgenes
\geometry{a4paper, margin=1in}

\begin{document}

\pagecolor{azulportada}
\begin{titlepage}
    \begin{tikzpicture}[remember picture, overlay]
        \node[anchor=north west, inner sep=10pt] at (current page.north west) {\includegraphics[width=1.5cm]{ICONO2i.png}};
    \end{tikzpicture}

    \vfill
    \begin{center}
        % Título
        \vspace{5cm}
        {\Huge \textbf{Memoria De Cálculo Estructural Para El Proyecto 'Casa Cojitambo'} \par}

        \vspace{15cm}
        {\Large MAMPRO SAS - Abril 2024 \par}
        
    \end{center}
    \vfill
\end{titlepage}

% Restablecer el color de fondo a blanco
\pagecolor{white}

\section{Introducción}
El presente proyecto estructural tiene como objetivo el diseño y análisis de una casa de un piso ubicada en la parroquia Cojitambo, en el cantón Azogues, Ecuador. Esta región, caracterizada por su belleza natural y su contexto geográfico particular, presenta desafíos específicos en términos de diseño y construcción para garantizar la seguridad y estabilidad de las estructuras ante las condiciones locales.

La casa de un piso se diseñará cumpliendo con las normativas y regulaciones de construcción ecuatorianas vigentes, garantizando la resistencia sísmica adecuada y considerando las cargas gravitacionales y de viento pertinentes para esta ubicación geográfica. Además, se utilizará el software ETABS en conjunto con Python, aprovechando su potencial para la modelización, análisis y optimización de estructuras, así como su capacidad para interactuar con la API de ETABS.

El desarrollo de este proyecto implica una cuidadosa planificación y un análisis detallado, con el fin de asegurar la integridad estructural y la durabilidad de la vivienda, proporcionando un entorno seguro y confortable para sus futuros ocupantes.

\subsection{Objetivos}
\begin{enumerate}[label = \alph*]
    \item Diseñar una estructura residencial segura y funcional que cumpla con las normativas y regulaciones de construcción ecuatorianas aplicables, garantizando la seguridad de los ocupantes.
    \item Optimizar el diseño estructural para maximizar la eficiencia en el uso de materiales y recursos, asegurando un balance adecuado entre costo, resistencia y durabilidad.
    \item Integrar consideraciones sísmicas específicas para la región de Cojitambo, utilizando técnicas de análisis avanzadas para evaluar y mitigar los efectos de los movimientos telúricos en la estructura.
    \item Implementar prácticas sostenibles de construcción que minimicen el impacto ambiental del proyecto, incluyendo la selección de materiales ecoamigables y soluciones de diseño que fomenten la eficiencia energética.
    \item Utilizar herramientas computacionales como el software ETABS y Python para modelar, analizar y optimizar la estructura, aprovechando las ventajas de la tecnología para mejorar la precisión y la eficacia del proceso de diseño.
    \item Colaborar estrechamente con otros profesionales involucrados en el proyecto, como arquitectos, ingenieros civiles y contratistas, para garantizar la coherencia y la integración de los aspectos estructurales con el diseño arquitectónico y los requisitos de construcción.
    \item Proporcionar documentación detallada y clara, incluyendo planos, cálculos y especificaciones técnicas, que sirva como guía para la construcción y el mantenimiento de la estructura a lo largo de su vida útil.
\end{enumerate}

\subsection{Normativa y Regulaciones Aplicables}
\begin{enumerate}
    \item  Diseñar una estructura residencial segura y funcional que cumpla con las normativas y regulaciones de construcción ecuatorianas aplicables, garantizando la seguridad de los ocupantes.

    \item Optimizar el diseño estructural para maximizar la eficiencia en el uso de materiales y recursos, asegurando un balance adecuado entre costo, resistencia y durabilidad.

    \item Integrar consideraciones sísmicas específicas para la región de Cojitambo, utilizando técnicas de análisis avanzadas para evaluar y mitigar los efectos de los movimientos telúricos en la estructura.

    \item Implementar prácticas sostenibles de construcción que minimicen el impacto ambiental del proyecto, incluyendo la selección de materiales ecoamigables y soluciones de diseño que fomenten la eficiencia energética.

    \item Utilizar herramientas computacionales como el software ETABS y Python para modelar, analizar y optimizar la estructura, aprovechando las ventajas de la tecnología para mejorar la precisión y la eficacia del proceso de diseño.

    \item Colaborar estrechamente con otros profesionales involucrados en el proyecto, como arquitectos, ingenieros civiles y contratistas, para garantizar la coherencia y la integración de los aspectos estructurales con el diseño arquitectónico y los requisitos de construcción.

    \item Proporcionar documentación detallada y clara, incluyendo planos, cálculos y especificaciones técnicas, que sirva como guía para la construcción y el mantenimiento de la estructura a lo largo de su vida útil.
\end{enumerate}

Por supuesto, es importante mencionar que además de la Norma Ecuatoriana de la Construcción NEC-15, se emplearán normativas internacionales reconocidas como la ACI 19 (American Concrete Institute) y la AISC (American Institute of Steel Construction), que proporcionan estándares y criterios ampliamente aceptados en la industria de la construcción. Algunos puntos a destacar sobre la utilización de estas normativas son:
\begin{enumerate}
    \item  \textbf{ACI (American Concrete Institute)}: Se utilizarán las especificaciones y criterios de diseño de la ACI 19 para el diseño de elementos de concreto, como columnas, vigas, losas y cimentaciones. 
    
    La ACI  proporciona recomendaciones detalladas sobre la resistencia del concreto, la disposición de refuerzos, los factores de seguridad y otros aspectos relevantes para garantizar la seguridad y durabilidad de las estructuras de concreto.
    
    \item \textbf{AISC (American Institute of Steel Construction)}: Se seguirán las normativas y estándares de la AISC para el diseño de elementos estructurales de acero, incluyendo vigas, columnas, conexiones y sistemas de arriostramiento.
    
    La AISC establece criterios para la selección de perfiles estructurales, el diseño de conexiones soldadas y atornilladas, así como otros aspectos relacionados con la resistencia y estabilidad de las estructuras de acero.
    Al integrar normativas como la ACI 19 y la AISC en el proceso de diseño y cálculo estructural, se garantiza que la estructura cumpla con estándares de calidad y seguridad reconocidos a nivel internacional, además de los requisitos locales establecidos por la Norma Ecuatoriana de la Construcción NEC-15. Esta combinación de normativas permite obtener diseños estructurales robustos y confiables que se adecuan a las condiciones y exigencias específicas del proyecto en Cojitambo, Azogues.
\end{enumerate}

\section{Descripción del diseño estructural}
faltaaaaaaaaaaaaaaaaa

\section{Cargas y Combinaciones}
La NEC15 establece que en general las contrucciones deberán diseñarse para resistir las 
combinacionesd de cargas permanentes, cargas variables y cargas accideantales.

%INICIA SECCION CARGAS, OBTENIDO CON FUNCIONES DE PYTHON

\subsection{Cargas Permanentes}
Las cargas permanentes consideradas para el proyecto son:
\begin{enumerate}
    \item Fibrocemento = 0.16 $KN/m^2$
    \item Acero en frio = 0.08 $KN/m^2$
    \item Acero en caliente = 1.2 $KN/m^2$
    \item Instalaciones = 1 $KN/m^2$
    \item Cielo Raso = 0.5 $KN/m^2$
    
\end{enumerate}


\end{document}